\documentclass[10pt]{TISmeetrapport}
\usepackage{natbib}
\usepackage{url}
\usepackage{listings}

\practicum{\LaTeX}
\nummerproef{1}
\naamproef{Meetrapport}
\studentauteur{S. Tudent}
\studentuitvoer{S. Tudent en T. Sudent}
\studentklas{NP2.a}
\datumuitvoer{2 mei 2017}
\datuminleveren{\today}
%\datuminleverenherkansing{\today}
\docent{D. Ocent}

\begin{document}
\makecoverpage
\abstract{Korte, bondinge samenvatting}
\section{Doel}
Wat moet er met de proef bepaald worden?

\section{Theorie}
Welke natuurkundige achtergrond is nodig om de proef te begrijpen? Welke formules (verbanden) gebruik je?

\section{Uitvoering}
De uitvoering bestaat uit twee delen. De ostelling zelf en daarna hoe je met de opstelling omgaat. Wat is je meetmethode?
\subsection{Opstelling}
Met behulp van een figuur de opstelling beschreven. Electrische schema's zijn netjes uitgewerkt en toegelicht. 
\subsection{Meetmethode}
Wat is er gemeten en hoe is de meting uitgevoerd? Wat is er nodig om het eindresultaat te behalen? Een opsomming is niet gewenst, de meetmethode moet in een (leesbare) tekstvorm genoteerd zijn.

\section{Resultaten}
De resulaten behaald met het proef zijn hier weergegeven. 
\subsection{Metingen}
Grafieken en kleine tabellen worden hier opgenemon, in bijlage(n) kunnen grotere tabellen weergegeven worden. Grote datasets worden niet in een meetrapport opgenomen, die zijn of digitaal beschikbaar gemaakt of op verzoek bij de schrijver(s) te krijgen. \\
Tabellen en figuren zijn voorzien van de juiste opmaak.

\subsection{Berekeningen}
Berekeningen zijn duidelijk en compact opgeschreven. Eventuele afleidingen kunnen in een bijlage opgenomen worden. Rekenvoorbeelden worden hier ook weergegeven. Foutenanalyse kan hier ook geplaatst worden.

\section{Conclusie}
Een korte samenvatting van de meetresultaten (einduitkomsten). Resultaten worden vergeleken met theoretische waarden en daar wordt commentaar op gegeven. 

\section{Vragen}
Expliciet gestelde vragen en opmerkingen kunnen hier beantwoord worden. 

\end{document}
